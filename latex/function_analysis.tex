\documentclass[12pt, a4paper]{article}
\usepackage[margin=2.0cm]{geometry}
\usepackage{stmaryrd}
\usepackage{amsmath}
\usepackage{amssymb}
\usepackage{enumerate}
\usepackage{bbold}
\usepackage{dsfont}
\usepackage{algorithm2e}
\usepackage{changepage}
\usepackage{undertilde}
\usepackage{hyperref}
\hypersetup{pdftex,colorlinks=true,allcolors=blue}
\usepackage{hypcap}
\usepackage{fancyhdr}
\usepackage{float}
\usepackage[english]{babel}
\usepackage{graphicx}
\usepackage{wrapfig}
\usepackage{multicol}
\usepackage[amsmath,hyperref]{ntheorem}
\usepackage{framed}
\usepackage{booktabs}
\usepackage{needspace}
\usepackage{qtree}

% Default fixed font does not support bold face
\DeclareFixedFont{\ttb}{T1}{txtt}{bx}{n}{10} % for bold
\DeclareFixedFont{\ttm}{T1}{txtt}{m}{n}{10}  % for normal

% Custom colors
\usepackage{color}
\definecolor{red}{rgb}{1,0,0}
\definecolor{deepblue}{rgb}{0,0,0.5}
\definecolor{deepred}{rgb}{0.6,0,0}
\definecolor{deepgreen}{rgb}{0,0.5,0}

\usepackage{listings}

% Python style for highlighting
\newcommand\pythonstyle{\lstset{
	language=Python,
	basicstyle=\ttm,
	otherkeywords={self},             % Add keywords here
	keywordstyle=\ttb\color{deepblue},
	emph={MyClass,__init__},          % Custom highlighting
	emphstyle=\ttb\color{deepred},    % Custom highlighting style
	commentstyle=\ttb\color{red},
	stringstyle=\color{deepgreen},
	frame=tb,                         % Any extra options here
	showstringspaces=false     
}}


% Python environment
\lstnewenvironment{python}[1][]
{
\pythonstyle
\lstset{#1}
}
{}

% Python for external files
\newcommand\pythonexternal[2][]{{
\pythonstyle
\lstinputlisting[#1]{#2}}}

% Python for inline
\newcommand\pythoninline[1]{{\pythonstyle\lstinline!#1!}}
\lstset{showstringspaces=false}

\newcommand{\thm}{
    \theoremstyle{plain}
    \theoremseparator{.}
    \theorembodyfont{\upshape}
    \theoremheaderfont{\bfseries}
}

\thm \newtheorem{definition}{Definition}[subsection]
\thm\newtheorem{proposition}[definition]{Proposition}
\thm\newtheorem{theorem}[definition]{Theorem}
\thm\newtheorem{lemma}[definition]{Lemma}
\thm\newtheorem{example}[definition]{Example}


\setlength{\parskip}{1em}
\setlength{\parindent}{0em}
\SetKwProg{Fn}{Function}{\string:}{}

\newcommand\tab[1][1cm]{\hspace*{#1}}

\pagestyle{fancy}
\fancyhf{}
\lhead{Dylan Cope}
\rhead{2016}

\begin{document}

	\begin{center}
	{\Large\bf Mathematical Interpreting}

	\vspace{.1in}
	Project Summary
	\vspace{.2in}

	\end{center}
	
	\section{Functions}
	
	Functions are converted in tree form for manipulation and evaluation, consider the function $f$,
	$$
		f(x, y) = \frac{4x^y - 2\sinh y}{9xy + \cos^2 \ln x}
	$$
	the equivalent function tree is,
	
	\Tree [.$f(x,y)$ [.$\div$ 
		[.$-$ 
			[.$\times$ 
				[.4 ] 
				[.pow [.$x$ ] [.$y$ ] ]
			] [.$\times$ 
				[.2 ]
				[.sinh [.$y$ ] ]
			]
		] [ .$+$ 
			[.$\times$ [.9 ] [.$\times$ [.$x$ ] [.$y$ ] ] ] 
			[.pow 
				[.cos [.ln [.$x$ ] ] ]
				[.2 ]
			]
		]
	] ]
	
	\section{Parsing}
	
	\section{Differentiation}
	
	For differentiable function $f, g, h$, with derivatives $f', g', h'$, we can derive patterns for the differentials
	of our operations of addition, subtraction, multiplication, division and exponentiation.
	\subsection{Addition/Subtraction}
	$$
		f = g \pm h \implies f' = g' \pm h'
	$$
	\subsection{Multiplication}
	\begin{align*}
		f &= gh \\
		\therefore f' &= g'h + gh'
	\end{align*}
	\subsection{Division}
	\begin{align*}
		f &= \frac{g}{h} \\
		&= g h^{-1} \\
		\therefore f' &= g' h^{-1} - g h' h^{-2} \\
		&= \frac{g'h - gh'}{h^2}
	\end{align*}
	\subsection{Exponentiation}
	\begin{align*}
		f &= g^{h} \\
		\therefore \ln f &= h \ln g \\
		\frac{f'}{f} &= h' \ln g + h \frac{g'}{g} \\
		f' &= g^h \Big(h' \ln g + h \frac{g'}{g}\Big)
	\end{align*}
	
	\section{Simplification}
	\begin{align*}
		0\times f &= f\times 0 = 0 \\
		1\times f &= f\times 1 = f \\
		0 + f &= f + 0 = f \\
		f - 0 &= f \\
		0 - f &= -1 \times f \\ 
		\frac{f}{1} &= f 
	\end{align*}
\end{document}